\documentclass[12pt]{report}

\usepackage[utf8]{inputenc}
\usepackage[T1]{fontenc}

\usepackage{listingsutf8}

\usepackage{graphicx}
\graphicspath{{img/}}

\title{Visible Light Communication System}
\author{Hugo}



\begin{document}

\maketitle

\tableofcontents



\chapter{Introduction}

This chapter is a presentation of Northumbria University as well as the researchers I have worked with, followed by an explanation of the advantages and drawbacks of visible light communication (VLC), and its most usual applications.

\section{About Northumbria University}

\subsection{Newcastle}

Northumbria University is located in Newcastle upon Tyne in the North East of England. Newcastle is one of the most populous cities of the UK, with 293,000 inhabitants. Newcastle has a massive amount of students, and two universities, Newcastle University and Northumbria University.

\subsection{Northumbria University}

Northumbria University is a former Polytechnic, established in 1969, and became the University of Northumbria in 1992. It employs more than 3,200 people and offers approximately 500 study programmes. It has two large campuses, City Campus East and City Campus West.

City Campus East is home to the Schools of Law, Design, and the Newcastle Business School. City Campus West is home to the Schools of Arts and Social Sciences, Built and Natural Environment, Computing, Engineering and Information Sciences and Life Sciences.

I worked in the Faculty of Engineering and Environment, more precisely in the Department of Mathematics, Physics and Electrical Engineering. This department focuses on a wide range of issues, from statistics to complex and nonlinear phenomena, astrophysics to smart materials, and communications to renewable energy.

\subsection{Optical communications research group}

My supervisor for this internship was Dr Hoa Le Minh. He is specialized in photonic networks, mobile and adhoc networks, smartphone technology, organic optical communications and, of course, optical wireless communications.
He is  is a member of the Optical Communications Research Group (OCRG), which is one of the few leading research groups in the UK and EU currently working on optical communications. In 2011, they presented their first experimental demonstration of a VLC link using organic LEDs, where they reached a potential data rate of 2.15 Mb/s.

Krishna, Sandeep, Zun ?



\section{Context}

\subsection{Existing data transfer technologies}

In the last years, the density of portable devices has grown drastically, as well as the quantity of data to transfer between them. This of course creates an ever-augmenting need for short-range networks between portable devices.  These can be designed using various technologies.

Direct electrical connections, using electrical wires or fiber optics have too many drawbacks : they are expensive, heavy, can break easily and, more importantly, are very unconvenient to use.

Wireless radio frequency communications, as used in 4G networks, also present several problems : RF circuitry is quite expensive and, more importantly, these channels are broadcast, which means that several ad hoc networks at the same location would be very hard to achieve wihtout many interferences. Moreover they require quite expensive licensing fees.

Wireless optical links, however, do not require RF circuitry but cheap LEDs. They can be easily contained and are generally short range so several networks can coexist in the same location quite easily. They have some limitations compared to RF links though. For example, for short range communications, mobility is very limited since they are stopped by any opaque boundary. Moreover, ambient noise is very intense since any source of light, including the sun, is a source of noise for the system.

\subsection{Application of VLC}

\subsubsection{Outdoor communication}

With autonomous vehicles being just around the corner, it is more and more important for vehicles to have a way to communicate with each other. VLC seems the most appropriate system to exchange data in automotive applications.
Indeed, LEDs are already used in traffic infrastructures and in the vehicle lightning system, which will greatly facilitate the integration of VLC technology. Moreover, since VLC is a line of sight technology, there would be no interferences between communication channels, even in high traffic density.

Wireless communications between vehicles could greatly improve security on the road. For example, safety messages could be transmitted from the traffic infrastructure to the vehicles and between the vehicles. Moreover, this system could increase the vehicle awareness of its surroundings or facilitate traffic flow.

\subsubsection{Indoor communication}

Indoor wireless optical communication is seen as complimentary to RF links rather than a replacement. Indeed, RF links will always allow for greater mobility. However, VLC really excels in very short-range and high-speed communication, for example between two devices in the same room, like a phone or a remote control, and an automated door or a speaker.



\section{Introduction to Visible Light Communications (VLC)}

This chapter introducts the concepts necessary to understand how a VLC system works.

\subsection{Architecture}

A VLC system consists of an emitter that modulates the light produced by a LED and a receiver that extract the modulated signal from the light detected by a photodiode. They are connected by a VLC channel which is a source of noise and attenuation.

\begin{figure}[h]
\centering
\includegraphics[scale=0.4]{system}
\end{figure}

\subsection{Channel}

The channel contains numerous sources of optical noise, like the sun, artificial lights, or even rain or snow. These, together with the low signal especially at long distances, affect the SNR.

Optical channels are however free from destructive interferences sometimes caused by reflection and refraction, unlike RF channels. Indeed, the antenna is way bigger than the wavelength of the signal. Temporal dispersion of the signal, however, remains a problem.

\subsection{Emitter}

The emitter transforms binary data into an electric signal encoding a frame of symbols, that can be sent using light. Moreover, since the light emitting device will often be used for other purpose, like illumination, the emitter must be capable to transfer data on top of a constant light, without inducing visible flickering.

Given that the frequency of visible light is of the order of 100-1000 THz, the standard modulations like frequency or phase modulation cannot be used. Instead, optical channels are intensity modulated. The optical intensity is the optical power emitted per solid angle (a solid angle can be seen as the generalisation of a 2D angle in 3D). An important constraint of optical channels is that the signal is an optical intensity and thus is nonnegative.

Mainly 2 types of light emitting devices are used for these applications.

-light emitting diode (LED): it is a standard junction diode and is usually emitting at around 850 nm.

-laser diode (LD): it produces an intense beam of light which is monochromatic, coherent and collimated.
LDs can operate at pulserates in the GHz range, while LEDs are limited to the MHz range. A LD ages faster and its characteristics change more with temperature. Because of this, a circuit is required to stabilize the LD. Finally, LDs can be dangerous for the eyes and filters must be used to diffuse the output light.

For most of the applications, LEDs are prefered for their simplicity and their price.

\subsection{Receiver}

The receiver transforms the light into an electrical signal then demodulate it. The receiver has the most influence on the performances of the VLC system because it determines the communication range. It is subject to many interferences caused by ambient lights so a filter is often used to discard the frequency that are not sent by the LED. Narrowing the field of view of the receiver is another possibilty to increase the SNR.

There are 2 types of photodiodes that can be used as a receiver:

-PIN photodiodes, with which the photocurrent is proportional to the optical power below 5 Ghz.

-Avalanches photodiodes (APD), which have a photocurrent greater than 1 because of the avalanche, of the order of $10^2$ to $10^4$, which allows for longer distances between the emitter and the detector. However, the avalanche creates shot noise. Moreover, the gain is non-linear thus additional circuitry is required. As APDs require higher voltage (at least 30V), they are not suitable for portable devices.

Most indoors links use PIN photodiodes. Outdoors links that require better sensivity may use APDs instead.



\chapter{Design of a VLC system}

The goal of this internship is to design a VLC system, made of an emitter and receiver, that could transport data (e.g. text) through visible light.
Concretely, the user should be able to type a text that would then be sent by the emitter to the receiver that would print it on a LCD.

\section{Simple data transfer}

The first thing to decide is how to transform data into an analog signal. In order to achieve this, a modulation scheme and a protocol must be chosen.

\subsection{Modulation}

The modulation used is Pulse Amplitude Modulation with 4 amplitude levels (4-PAM). It is one of the most widely used modulation for visible light communication. It is quite easy to implement, and is two to four times faster than BPSK (e.g. Manchester coding), albeit less robust because no clock recovery is possible.

With this modulation, there are 4 possible symbols, each representing 2 bits.

\begin{center}
\begin{tabular}{c|c|c}
Symbol&Bits&Value\\
\hline
0&00&-3\\
1&01&-1\\
2&10&+3\\
3&11&+1\\
\end{tabular}
\end{center}

\begin{figure}[h]
\centering
\includegraphics[scale=0.6]{PAM4}
\end{figure}

It is important to use Gray Code because the only way message integrity is checked is by computing the parity bits.

\subsection{Protocol}

There is a need for a protocol in order to indentify the beginning and the end of the message, and to check the parity values.

Example of an "Hello" message:


For hardware reasons, the signal frequency is limited to 100Hz, thus this message would take 616ms to be sent.

\begin{center}
\begin{tabular}{ccccc|l}
0&0&0&0&0&No message\\
0&1&2&3&2&Preamble\\
3&3&3&3&3&Start of Frame\\
1&0&2&0&0&"H"\\
1&2&1&1&0&"e"\\
1&2&3&0&0&"l"\\
1&2&3&0&0&"l"\\
1&2&3&3&0&"o"\\
0&0&0&0&2&End of Frame\\
0&0&0&0&0&No message\\
\end{tabular}
\end{center}

The stream is divided into frames, which are divided in words of 5 symbols (10 bits). A frame can contain by design a maximum of 16 bytes of data.

\subsubsection{Preamble}

Each frame starts with a preamble: 01232. It is a fixed word which let the receiver synchronize with the signal phase.

\subsubsection{Start of frame}

Then there is a start of frame (SOF): 33333. It is a fixed word which marks the beginning of the message. The SOF is the only word that has a 3 in last position, thus is the only location in the message where there can be five 3 in a row, making the detection of the beginning of the message very straightforward.

\subsubsection{Payload}

Then there is the payload, which is composed of the message, intercalated with parity bits. More precisely, each word is composed of 4 symbols which form a byte, and a 5th symbol, that acts as a parity bit by taking the values 0 or 1 in such a way that there is always an even number of bits "1" in a word.

For example, "H" is 01 00 10 00. There already is an even number (2) of "1", so we add 00 at the end, all of which, translated into symbols, gives 10200.

\subsubsection{End of frame}

Finally, there is an End of Frame (EOF): 00002. It is a fixed word which, contrary to the payload, ends with a 2 and marks the end of the message.



\section{Design of the emitter}

\subsection{Message input}

Keypad

\subsection{Generation of the signal}

?

\subsection{Output filtering}

As the output is generated by PWM, it must be filtered before being sent to the diode in order to remove the variations created by the PWM. A simple low pass RC filter is used for this.
The resistor is $10k\Omega$



\section{Design of the receiver}

Most of the photoreceiver on the market are designed to be used in laboratory. They are of very good quality, but also big and expensive. The goal is to design a little receiver that could be integrated to the existing circuit.
For this, we will be using an Arduino nano programmed in C.

\subsection{Transimpedance amplifier}

At the center of an optical receiver is a photodiode. The problem is that a photodiode gives a current proportional to the illumination, and currents are not as easy to manipulate as voltages. The solution is to put a transimpedance amplifier after the diode.

A transimpedance amplifier converts a current into a voltage. It is a very simple circuit, composed only of an operational amplifier and a resistor.

\begin{figure}[h]
\centering
\includegraphics[scale=0.7]{transimpedance_amplifier}
\end{figure}

Here, when $V_b = 0$, $V_{out}$ is proportional to $I_p$ and the relation between them is $$V_{out} = - R_f \times I_p$$
In order to have a positive output, the photodiode could be put the other way.
The capacitor$ C_f$ is needed because the photodiode can be seen as a current source and a capacitor. Thus, in order to have a stable circuit, a capacitor $C_f$ is needed.

The value of $R_f$ determines the gain of the circuit. While the value of $C_f$ determines its bandwidth. To determine a correct value of $C_f$, a good approximation is $$C_f \times R_f = C_{in} \times R_{in}$$ where $C_{in}$ and $R_{in}$ are the input capacitance and resistance of the operational amplifier.

With a light emitting a few thousands lux towards the receiver, in order to get an output between 0V and 5V, a $R_f = 1M\Omega$ is well suited.
 The operational amplifier used is a LM741 with $C_{in} = 10pF$ and $R_{in} = 1M\Omega$. Thus, in order to have a working circuit at around 1kHz, we take $C_f = 10pF$.
With these values, the output is around 4V. ?

\subsection{DC centering}

?

\subsection{Message output}

?



\chapter{Analysis of the internship}

Boring or not boring ?



\appendix

\chapter{Arduino code}

\section{Emitter}

\subsection{loop.cpp}

\lstinputlisting{../emitter/loop.cpp}

\subsection{define.h}

\lstinputlisting{../emitter/define.h}

\subsection{signal.h}

\lstinputlisting{../emitter/signal.h}

\subsection{signal.cpp}

\lstinputlisting{../emitter/signal.cpp}

\subsection{tools.h}

\lstinputlisting{../emitter/tools.h}

\subsection{tools.cpp}

\lstinputlisting{../emitter/tools.cpp}

\section{Receiver}

\subsection{loop.cpp}

\lstinputlisting{../project/loop.cpp}

\subsection{define.h}

\lstinputlisting{../project/define.h}

\subsection{signal.h}

\lstinputlisting{../project/signal.h}

\subsection{signal.cpp}

\lstinputlisting{../project/signal.cpp}

\subsection{tools.h}

\lstinputlisting{../project/tools.h}

\subsection{tools.cpp}

\lstinputlisting{../project/tools.cpp}



\begin{thebibliography}{9}

\bibitem{wocs}
Steve Hranilovic.
\textit{Wireless Optical Communication Systems.}
Springer, 2004.

\bibitem{these}
Alin Cailean.
\textit{Etude et réalisation d’un système de communications par lumière visible (VLC/LiFi). Application au domaine automobile.}
Université de Versailles Saint-Quentin en Yvelines, 2014.

\bibitem{pam4}
Winston Way.
\textit{PAM-4: A Key Solution For Next-Generation Short-Haul Optical Fiber Links.}
2015.

\end{thebibliography}

\end{document}
